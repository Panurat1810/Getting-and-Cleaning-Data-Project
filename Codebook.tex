% Options for packages loaded elsewhere
\PassOptionsToPackage{unicode}{hyperref}
\PassOptionsToPackage{hyphens}{url}
%
\documentclass[
]{article}
\usepackage{amsmath,amssymb}
\usepackage{lmodern}
\usepackage{iftex}
\ifPDFTeX
  \usepackage[T1]{fontenc}
  \usepackage[utf8]{inputenc}
  \usepackage{textcomp} % provide euro and other symbols
\else % if luatex or xetex
  \usepackage{unicode-math}
  \defaultfontfeatures{Scale=MatchLowercase}
  \defaultfontfeatures[\rmfamily]{Ligatures=TeX,Scale=1}
\fi
% Use upquote if available, for straight quotes in verbatim environments
\IfFileExists{upquote.sty}{\usepackage{upquote}}{}
\IfFileExists{microtype.sty}{% use microtype if available
  \usepackage[]{microtype}
  \UseMicrotypeSet[protrusion]{basicmath} % disable protrusion for tt fonts
}{}
\makeatletter
\@ifundefined{KOMAClassName}{% if non-KOMA class
  \IfFileExists{parskip.sty}{%
    \usepackage{parskip}
  }{% else
    \setlength{\parindent}{0pt}
    \setlength{\parskip}{6pt plus 2pt minus 1pt}}
}{% if KOMA class
  \KOMAoptions{parskip=half}}
\makeatother
\usepackage{xcolor}
\usepackage[margin=1in]{geometry}
\usepackage{graphicx}
\makeatletter
\def\maxwidth{\ifdim\Gin@nat@width>\linewidth\linewidth\else\Gin@nat@width\fi}
\def\maxheight{\ifdim\Gin@nat@height>\textheight\textheight\else\Gin@nat@height\fi}
\makeatother
% Scale images if necessary, so that they will not overflow the page
% margins by default, and it is still possible to overwrite the defaults
% using explicit options in \includegraphics[width, height, ...]{}
\setkeys{Gin}{width=\maxwidth,height=\maxheight,keepaspectratio}
% Set default figure placement to htbp
\makeatletter
\def\fps@figure{htbp}
\makeatother
\setlength{\emergencystretch}{3em} % prevent overfull lines
\providecommand{\tightlist}{%
  \setlength{\itemsep}{0pt}\setlength{\parskip}{0pt}}
\setcounter{secnumdepth}{-\maxdimen} % remove section numbering
\ifLuaTeX
  \usepackage{selnolig}  % disable illegal ligatures
\fi
\IfFileExists{bookmark.sty}{\usepackage{bookmark}}{\usepackage{hyperref}}
\IfFileExists{xurl.sty}{\usepackage{xurl}}{} % add URL line breaks if available
\urlstyle{same} % disable monospaced font for URLs
\hypersetup{
  pdftitle={Codebook},
  pdfauthor={me},
  hidelinks,
  pdfcreator={LaTeX via pandoc}}

\title{Codebook}
\author{me}
\date{2023-07-07}

\begin{document}
\maketitle

The run\_analysis.R script performs the data preparation and then
followed by the 5 steps required as described in the course project's
definition.

\hypertarget{download-the-dataset}{%
\subsubsection{1.Download the dataset}\label{download-the-dataset}}

Dataset downloaded and extracted under the folder called UCI HAR Dataset

\hypertarget{assign-each-data-to-variables}{%
\subsubsection{2.Assign each data to
variables}\label{assign-each-data-to-variables}}

features \textless- features.txt : 561 rows, 2 columns\\
activities \textless- activity\_labels.txt : 6 rows, 2 columns~
subject\_test \textless- test/subject\_test.txt : 2947 rows, 1 column\\
x\_test \textless- test/X\_test.txt :2947 rows, 561 columns\\
y\_test \textless- test/y\_test.txt : 2947 rows, 1 columns\\
subject\_train \textless- test/subject\_train.txt : 7352 rows, 1
column\\
x\_train\textless- test/X\_train.txt : 7352 rows, 561 columns~ y\_train
\textless- test/y\_train.txt : 7352 rows, 1 columns\\

\hypertarget{merges-the-training-and-the-test-sets-to-create-one-data-set}{%
\subsubsection{3.Merges the training and the test sets to create one
data
set}\label{merges-the-training-and-the-test-sets-to-create-one-data-set}}

X (10299 rows, 561 columns) is created by merging x\_train and x\_test
using rbind() function\\
Y (10299 rows, 1 column) is created by merging y\_train and y\_test
using rbind() function\\
Subject (10299 rows, 1 column) is created by merging subject\_train and
subject\_test using rbind() function\\
Merged\_Data (10299 rows, 563 column) is created by merging Subject, Y
and X using cbind() function\\

\hypertarget{extracts-only-the-measurements-on-the-mean-and-standard-deviation-for-each-measurement}{%
\subsubsection{4.Extracts only the measurements on the mean and standard
deviation for each
measurement}\label{extracts-only-the-measurements-on-the-mean-and-standard-deviation-for-each-measurement}}

TidyData (10299 rows, 88 columns) is created by subsetting Merged\_Data,
selecting only columns: subject, code and the measurements on the mean
and standard deviation (std) for each measurement

\hypertarget{uses-descriptive-activity-names-to-name-the-activities-in-the-data-set}{%
\subsubsection{5.Uses descriptive activity names to name the activities
in the data
set}\label{uses-descriptive-activity-names-to-name-the-activities-in-the-data-set}}

Entire numbers in code column of the TidyData replaced with
corresponding activity taken from second column of the activities
variable

\hypertarget{appropriately-labels-the-data-set-with-descriptive-variable-names}{%
\subsubsection{6.Appropriately labels the data set with descriptive
variable
names}\label{appropriately-labels-the-data-set-with-descriptive-variable-names}}

code column in TidyData renamed into activities\\
All Acc in column's name replaced by Accelerometer\\
All Gyro in column's name replaced by Gyroscope\\
All BodyBody in column's name replaced by Body\\
All Mag in column's name replaced by Magnitude\\
All start with character f in column's name replaced by Frequency\\
All start with character t in column's name replaced by Time\\

\hypertarget{from-the-data-set-in-step-4-creates-a-second-independent-tidy-data-set-with-the-average-of-each-variable-for-each-activity-and-each-subject}{%
\subsubsection{7.From the data set in step 4, creates a second,
independent tidy data set with the average of each variable for each
activity and each
subject}\label{from-the-data-set-in-step-4-creates-a-second-independent-tidy-data-set-with-the-average-of-each-variable-for-each-activity-and-each-subject}}

final\_data (180 rows, 88 columns) is created by sumarizing TidyData
taking the means of each variable for each activity and each subject,
after groupped by subject and activity. Export final\_data into
FinalData.txt file.

\end{document}
